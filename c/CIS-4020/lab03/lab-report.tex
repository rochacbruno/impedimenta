%%%%%%%%%%%%%%%%%%%%%%%%%%%%%%%%%%%%%%%%%%%%%%%%%%%%%%%%%%%%%%%%%%%%%%%%%%%%%%%%
% FILE    : lab-report.tex
% SUBJECT : CIS-4020 Lab 2 report
% AUTHOR  : Jeremy Audet <ichimonji10 at gmail dot com>
% LICENSE : Public Domain
%%%%%%%%%%%%%%%%%%%%%%%%%%%%%%%%%%%%%%%%%%%%%%%%%%%%%%%%%%%%%%%%%%%%%%%%%%%%%%%%

\documentclass{article}

\usepackage{fancyvrb}
\usepackage[pdftex]{graphicx}
\usepackage{listings}
\usepackage{hyperref}
\usepackage{url}

% The following are settings for the listings package.
% See the listings package documentation for more information.
\lstset{
    language=C,
    basicstyle=\small,
    stringstyle=\ttfamily,
    commentstyle=\ttfamily,
    xleftmargin=0.25in,
    showstringspaces=false
}

\begin{document}

\title{CIS-4020 Lab \#3\\TITLE}
\author{Jeremy Audet\thanks{Ichimonji10@gmail.com}\\
    Vermont Technical College}
    \date{October 08, 2013}
    \maketitle

\begin{abstract}
The purpose of this lab is to rewrite the scheduler for the Phoenix operating
system.
\end{abstract}

\section{Introduction}
\label{sec:introduction}

Phoenix is a small micro-kernel operating system created by three VTC students
as a senior project. Due to it's small size, it is relatively easy to understand
and manipulate. Phoenix is capable of switching between multiple processes, and
the process switching logic is very simple. Each time a scheduling interrupt
occurs, Phoenix searches through a ringbuffer containing process information and
runs the next viable process.

In this lab, the Phoenix scheduler is rewritten to support process priority
levels.

\section{Procedure}
\label{sec:procedure}

\begin{enumerate}
\item Edit \texttt{src/kernel/xthread.h}. Create a new type called
\texttt{priority\_t}, then add members \texttt{priority} and \texttt{score} to
\texttt{struct process}.
\begin{lstlisting}
typedef enum {LOW, NORMAL, HIGH} priority_t;
...
typedef struct {
    word       far * stack; // location of stack
    bool       runnable; // true if thread should be run by scheduler
    processID  pid; // unique identifier
    priority_t priority; // used by scheduler
    int        score; // used by scheduler
} process;
\end{lstlisting}

\item Edit \texttt{src/kernel/xtimer.c}. Rewrite function \texttt{Schedule} to
take advantage of the new struct members. Decide that both function
\texttt{Schedule} and the backing functions declared in
\texttt{src/kernel/xrndbuff.h} are unacceptably ugly and should be rewritten
from scratch. Rewrite them.
\end{enumerate}

\section{Discussion}
\label{sec:discussion}

The process scheduler relies heavily on the functions declared in
\texttt{src/kernel/xrndbuff.h} to determine which process should run next.
Unfortunately, those functions are ill-documented and over-engineered. For
example, the documentation for \texttt{get\_next()} states:

\begin{quote}
This function returns a pointer to the next process that is defined in
the round buffer. As a side effect it also records this process for use by
\texttt{set\_current()}.
\end{quote}

Nowhere is a conceptual model for how to use the internal variables
\texttt{current} or \texttt{next} laid out. Nor is it a good thing for
side-effects to be designed into a system: the programmer should not need to
mentally track invisible side effects. Issues such as these make the existing
roundbuffer implementation unacceptable.

The solution I designed resolves these issues by providing more thorough
documentation and reducing the prevalence of side-effects. The solution I
designed also attempts to break complex actions down into more granular actions.
For example, if the user wishes to find the next viable process after the
current process, they must give \texttt{get\_next()} a process ID for use as a
starting point in it's search, and this can be provided by either
\texttt{get\_current()}. Formerly, \texttt{get\_next()} would simply find the
next process after the current process, and no starting point could be provded.

See \texttt{xrndbuff.h}\ref{fig:xrndbuff.h} for more details.

Several other issues were encountered while accomplishing this lab. The
procedure for compiling Phoenix and starting it with Bochs was labor-intensive,
requiring the user to type out multiple commands. This procedure has been
simplified by the creation of a script, \texttt{src/makeandrun.sh}.

It was also discovered that the existing makefiles do not properly clean Phoenix
after each build. Though \texttt{wmake} and the makefiles should be able to
detect which files have been changed and recompile those files as necessary,
this was not the case. This was the cause of many incorrect assumptions and
frustrations during development. The solution was to update the makefiles to
clean a greater set of files and ensure that \texttt{src/makeandrun.sh} always
called \texttt{wmake clean}.

The scheduling system which was devised prevents lower-priority threads from
starving while still letting higher-priority threads run more often. Each time
the scheduler is called, each process accrues a certain number of "points" based
on it's priority, and higher-priority processes accrue points faster than
lower-priority processes. The scheduler picks the highest-scoring process,
resets it's score to zero, and then runs it. The docstrings for the
\texttt{Schedule}\label{fix:Schedule} function contains details.

The end results are as follows:

\begin{enumerate}
\item Phoenix runs acceptably fast. With the original codebase, programs 0 and 1
ran at at 90 and 64 IPS, respectively. With the updated codebase, they run at 80
and 71 IPS, respectively. It should be noted that the author does not really
know what IPS means, and that these tests were completely informal.
\item Unfortunately, Phoenix experiences degraded functionality. Test program 0
no longer works correctly with the updated codebase.
\end{enumerate}

\section{Conclusion}
\label{sec:conclusion}

In this lab, I examined how the Phoenix micro-kernel operating system keeps
track of process information and schedules those processes. I rewrote both of
those pices of functionality. The importance of correct and proper tools, such
as nicely-maintained makefiles, helper scripts and automated tests was
reinforced.

\section{Appendix}
\label{sec:appendix}

\begin{figure}[tbhp] % place fig top, bottom, here, or alone on page
\label{fig:xrndbuff.h}
\begin{lstlisting}[
    frame=single,
    caption={Process Ringbuffer},
    xleftmargin=0in
]
/*! \\file xrndbuff.h Circular buffer holding process information.
 *
 * A ringbuffer is used to store information about processes, and the functions
 * defined here are used to interact with that ringbuffer. The ringbuffer is of
 * a fixed size, and each process fills one element of that ringbuffer.
 * Therefore, not all elements of the ringbuffer are filled with useful
 * information at all times.
 *
 * `add_process()` should be called before any other function, or else undefined
 * behaviour will occur. When `add_process()` is called for the first time, an
 * index is initialized which points to the location of the new process in the
 * ringbuffer. Subsequent calls to `add_process()` do not affect the index.
 *
 * Only `add_process()` and `set_current()` affect the index. See the function
 * definitions for more details.
 */

#ifndef XROUNDBUFF_H
#define XROUNDBUFF_H

#include "xthread.h"

int  add_process( process * );
bool set_current( processID );

process *get_current( );
process *get_process( processID );
process *get_next( processID );

#endif
\end{lstlisting}
\end{figure}

\begin{figure}[tbhp] % place fig top, bottom, here, or alone on page
\label{fig:xrndbuff.h}
\begin{lstlisting}[
    frame=single,
    caption={Process Ringbuffer},
    xleftmargin=0in
]
//! Decide which process should run next and return it's stack. Timer interrupt.
/*!
 * The following logic is used to select the next process:
 *
 * 1. Increment the score of each runnable process. The greater a process's
 *    priority, the greater the increment.
 * 2. Select the process with the highest score. If multiple processes have the
 *    same score, select the first process after the "current" process, where
 *    the "current" process is the process pointed to by the ringbuffer index.
 * 3. Make the selected process the "current" process, set it's score to zero
 *    and run it.
 *
 * Note that only runnable processes are given points. If non-runnable processs
 * are given points, they will attain insane scores. This could be problematic
 * if numerous non-runnable processes all became runnable at the same time
 *
 * This function is inefficient: the score of *every* process is bumped whenever
 * the function is called. As MAX_THREADS grows, the problem will only get
 * worse. Better solutions should be considered. One possible solution is to
 * examine only a limited number of processes each time this interrupt occurs,
 * rather than examining all processes.
 */
word far *Schedule( word far *p )
{
    process * const current = get_current( );
    process * candidate = get_current( ); // Candidate for next runnable process
    process * choice; // Choice for next runnable process
    processID idle; // Used to find the idle thread

    if( NULL == current ) {
        print_at( count++, col, "thread pointing to NULL", 0x04 );
        print_at( count++, col, "(has add_process() been called?)", 0x04 );
        return p;
    }
    // What does this do? It is necessary for test program 1 to function.
    current->stack = p;

    // Make a default choice...
    idle.pid = IDLE;
    choice = get_process( idle );
    // ... then see if any other suitable candidates exist.
    do {
        candidate = get_next( candidate->pid );
        if( true == candidate->runnable ) {
            // Bump score before considering a process. This ensures that
            // `candidate` will always out-score the idle thread.
            candidate->score += candidate->priority;
            if( candidate->score > choice->score ) { choice = candidate; }
        }
    } while( candidate->pid.pid != current->pid.pid );

    choice->score = 0;
    set_current( choice->pid ); // update ringbuffer index
    return choice->stack;
}
\end{lstlisting}
\end{figure}

\end{document}
